% vim:ft=tex
% rubber: module xelatex
\subsection{Face feature extraction}
\label{sec:features-prior}

Feature identification and extraction works by finding image "interest points" in the form of various basic mathematical features, including corners, blobs, edges, ridges, and gradients. A wide variety of algorithms support the identification and extraction of such features. Extracting features from face images is one specific application of interest.

The first scale-invariant feature extraction algorithm was SIFT, described in \cite{SIFT}. SIFT is designed to reliably detect features in altered (e.g. contrast-shifted, re-oriented or scaled) images. SIFT uses a database of training images with identified features to attempt to classify objects in new images based on their feature vectors. Image features are assumed to be the maxima and minima of the difference of Gaussians applied to the images. Outliers are discovered and removed by comparing each image feature with the model, verified by least squares. The SURF extractor, described in \cite{SURF}, is an attempt to improve upon SIFT by speeding up computation without loss of computation quality or robustness. SURF is meant to achieve this improvement by convolving images with integral images, using a Hessian matrix-based detector, using a distribution-based descriptor, and simplifying to only 64 dimensions (see \cite{SURF}).

There are many other algorithms in the modern `state of the art'. For example, \cite{speechreading} describes a comprehensive and robust face feature acquisition system capable of dealing with variations in pose, lighting, and noise, with the aim of modelling mouths for purposes such as automatic speechreading. \cite{discriminantanalysis} describes automatic face recognition based on linear discriminant analysis of spatial and wavelet aspects of human faces. Such analysis provides a platform for objective measurement of how \emph{significant} any given visual information is for identifying human faces. Another innovation in the field is the use of 2D image matrices rather than 1D vectors in principal component analysis (PCA). \cite{2d-pca} describes the results as providing both higher-quality recognition and more efficient computation than single-dimensional PCA.
