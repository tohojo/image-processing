% vim:ft=tex
% rubber: module xelatex
\subsection{Distortion removal}
\label{sec:distortion-prior}

Lens distortion is an issue in many fields, from the more theoretical to the practical (such as film production). According to \cite{postproduction}, it may be desirous not only to eliminate lens distortion, but also to \emph{alter} it (e.g. when combining footage from different cameras) or even \emph{introduce} it (e.g. when integrating computer graphics into footage).

In practise, barrel distortion is the most common type of distortion, commonly introduced by wide-angle or low-quality lenses (\cite{algebraic-distortion}). There are various methods of modelling lens distortion, not all of which are appropriate for all types of lens. To date, most lens distortion correction algorithms have either used information about the 3D co-ordinates of points in the image (such as \cite{Tsai}), or used as their basis the quasi-tautological principle that straight lines in an image should be straight (see \cite{straightlines}). However, some researchers have proposed algorithms based on other theoretical grounds. For instance, \cite{wide-angle} describes the correction of major distortion in super-wide angle and fish-eye lenses through \emph{circle fitting}.

\cite{algebraic-distortion} describes a mathematical model of radial lens distortion estimating the lens distortion parameters of a camera (or image), and gives an accompanying correction algorithm based on the rectification of lines in the images. The method is similar to the fundamental process described in e.g. \cite{straightlines}, but with some innovations. Essentially, the technique is an optimising process which determines the undistorted coefficients by minimising the error between the input image and its most relevant radial distortion model. The algorithm finds the lens distortion parameters by minimizing a 4 total-degree polynomial in several variables. The mathematical theory behind their implementation (for a full explanation of which, see \cite{algebraic-distortion}) leads the authors to only consider coefficients $\kappa_{0}$, $\kappa_{2}$, and $\kappa_{4}$. The IPOLdistortion algorithm functions by setting the first distortion parameter to 1 (to avoid the trivial solution $\kappa_{0}$ = 0), and then minimising a distortion error measure function in the form of an energy function of the authors' design. This function is a real-valued polynomial in the variable $\kappa$. To minimise the function, IPOLdistortion finds the solutions to the algebraic system of equations generated by its gradient. The implementation also introduces a "zoom factor" minimising the distance between distorted and corrected points, in order to to create corrected points as close as possible to the distorted ones \cite{algebraic-distortion}.
