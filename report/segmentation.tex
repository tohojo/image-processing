% vim:ft=tex
% rubber: module xelatex
\subsection{Image segmentation}
\label{sec:segmentation}
We have implemented a simple thresholding algorithm, and a split and
merge image segmentation algorithm from the description on the lecture
slides. Key points:

\begin{itemize}

\item The thresholding algorithm splits pixels into background and
  foreground on a threshold which is the mean of the pixel
  intensities. Optionally, this can be adapted iteratively by
  adjusting the threshold to be halfway between the means of the upper
  and lower half, until the threshold converges. No attempt is made to
  figure out which parts of the image are background and which are
  foreground. Instead, a user parameter sets whether the background in
  the original image is light or dark.
\item Regions in the split and merge algorithm are represented using
  the Region class, which unfortunately tries to be a bit too clever,
  and so represents regions using the points of their border. This
  works well for convex regions, but breaks down when regions have
  weird shapes. This means that because the convexity assumption
  breaks down, in some cases regions that could have been merged will
  not be, since pixels will be included in checks that should not. In
  general it makes the segmentation less than perfect. Solution:
  Actually store the pixels that make up the region.
\item In the merge step of the split and merge segmentation, regions
  are matched to each other rather naively (i.e. all regions are
  matched to each other). This makes the implementation rather slow
  when there are more than a few hundred regions after the split step.
  Solution: Only merge regions that are neighbouring (this will be
  easier given a better region representation).
\end{itemize}
