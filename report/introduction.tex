% vim:ft=tex
% rubber: module xelatex

\section{Introduction}

This work is motivated by the aim of eventually creating a 3D vision system. To this end, we built an application with functionality covering four processes: image segmentation, face feature extraction, camera calibration and distortion removal. Our program is written in C++ using the Qt library for its GUI, the OpenCV library for basic image manipulation functions, and several other libraries for specific processing functions.

This report is laid out as follows. In section~\ref{sec:prior}, we discuss prior work in the field. In section~\ref{sec:prog}, we discuss our program, implementation and experimental results. This is divided into discussions of segmentation in section~\ref{sec:segmentation}, feature extraction in section~\ref{sec:features}, camera calibration in section~\ref{sec:calibration}, and distortion removal in section~\ref{sec:distortion}. We lay out our findings in those sections, and then briefly summarise our conclusions in section~\ref{sec:conclusions}.

Contributions to this work are divided between Toke Høiland-Jørgensen, whose focus was image segmentation and face feature extraction, and who was additionally responsible for the underlying interface and structure of the code; and Ben Meadows, whose focus was camera calibration and distortion correction, and who was additionally responsible for the final editing of the jointly-written report. Both had input on all the key tasks at each stage of the work.
