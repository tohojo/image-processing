% vim:ft=tex
% rubber: module xelatex

\section{Introduction}
This report documents the second part of our efforts to create a computer vision
system capable of doing face recognition from images taken by a 3D camera. The
first part of the report documented the work we have done on image segmentation,
feature extraction, camera calibration and distortion removal. This part
documents the work we have done on image rectification, stereo matching and face
recognition, some of which builds on the work from the first part.

This report is laid out as follows. In section~\ref{sec:improvements}, we make
several improvements and additions to our last project for this course. In
section~\ref{sec:prior}, we discuss prior work in the field. In
section~\ref{sec:prog}, we discuss our program, implementation and experimental
results. This is divided into discussions of rectification in
subsection~\ref{sec:rectificatoin}, stereo matching in
subsection~\ref{sec:stereo}, and face recognition in
subsection~\ref{sec:face-rec}. We give our findings in those subsections, and
then briefly summarise our conclusions in section~\ref{sec:conclusions}.

Contributors to this work are Toke Høiland-Jørgensen, whose focus has been
rectification and preprocessing of images for the face recognition and
Ben Meadows, whose focus has been stereo matching and PCA.
