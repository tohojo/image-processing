% vim:ft=tex
% rubber: module xelatex
\subsection{Rectification}
\label{sec:rectification-prior}

Rectification is the modification of a pair of stereo images so that the
epipolar lines of the stereo geometry are horizontal in both images, and so that
scanlines correspond to each other (i.e. so that the same physical points are at
the same y coordinates in the two images). Rectification can be defined in terms
of the intrinsic and extrinsic camera parameters, but doing so means they have
to be known (i.e. the cameras need to be calibrated). In cases where this is not
possible, several approaches to rectification without calibration have been
proposed.

\citet{chen03} propose a rectification algorithm using the fundamental matrix,
without requiring calibration of the cameras. Another uncalibrated matching
based on feature point extraction of grey-scale images is proposed by
\citet{papadimitriou96:_epipol}; this approach works up to a 10\% angle
difference between the two images.

Another uncalibrated approach that uses a three-stage approach (projective
transform, similarity transform, shearing transform) to minimise the distortion
of the rectified images to ``a well-defined minimum'' is given by
\citet{loop99:_comput}. Finally, several calibration methods for trinocular
images (i.e. with three cameras) are given by \citet{sun03:_uncal}.

