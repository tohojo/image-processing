% vim:ft=tex
% rubber: module xelatex
\subsection{Stereo matching}
\label{sec:stereo-prior}

% TODO: Review of this section is probably needed.

Stereo matching, the process of reconstructing a 3D scene from a pair of stereo
images, is an important part of computer vision. Various approaches to the
problem have been proposed.

Stereo matching algorithms typically fall into two categories: those doing local
matching (i.e. per-pixel or in pixel neighbourhoods) and those using a global
matching model (i.e. analysing the whole image). The global method requires
keeping track of a lot of information about the whole image, which can be quite
demanding on computational resources (mainly memory), while the local method
requires selecting an appropriate matching window, which may vary between
images. An evaluation of a large number of different algorithms is given by
\citet{scharstein02:_taxon_evaluat_dense_two_frame}.

Examples of local matching approaches include a correlation-based matching
algorithm with a fast GPU implementation given by \citet{weber09}. Global
matching approaches include the use of graph cuts
\cite{kolmogorov01:_comput}, and belief propagation using Markov fields
\cite{felzenszwalb06:_effic_belief_propag_early_vision}.

Speeding up the speed of stereo matching is the focus of
\citet{geiger11:_effic_large_scale_stereo_match}, who achieve near-realtime
matching speeds with high accuracy using a set of support points that can be
accurately identified between the two images. Another realtime approach is taken
by \citet{realtimestereo}, by the application of dynamic programming techniques.

%%%% Couldn't find this article, so omitting for now %%%%
% Stereo matching using block matching. e.g. by Kurt Konolige. A very fast
% one-pass stereo matching algorithm. Matches blocks or windows of the image, not
% individual pixels.

