% vim:ft=tex
% rubber: module xelatex
\section{Our program}
\label{sec:prog}
Our program is written in C++ using the Qt and OpenCV libraries. The application
consists of a Qt-based GUI that allows the user to load images, select between
various image processors, select parameters and peruse the results of the
processing by zooming and panning on the output image. Furthermore, it is
possible to select points of interest (POIs) by double clicking on the image,
which can be used to select points for the calibration algorithm. The GUI also
has a log output window for textual output from the algorithms. It is possible
to perform every step of functional face recognition, from the calibration that
provides the rectification data to the final PCA analysis, within the system we
have built.

Each processor is implemented as a class that specifies which parameters are
available to it. The parameters can be set by the user with the help of the Qt
property system and the QPropertyEditor library. After parameterisation, the
actual processing is performed in a separate thread, keeping the interface
responsive and making it possible for the user to cancel a long-running
processor.

% vim:ft=tex
% rubber: module xelatex
\subsection{Rectification}
\label{sec:rectification}



% vim:ft=tex
% rubber: module xelatex

\subsection{Stereo matching}
\label{sec:stereo}

We followed the methods described in \cite{realtimestereo} to implement dynamic programming for stereo matching.


\begin{comment}
PSEUDOCODE:

A[i,j] = Minimum + Dif(ColorR,ColorL)
At the total beginning, A[0,0] has to be initialized with 0. Afterwards, all others elements are evaluated in the order from the upper left to the lower right corner.
It is absolutely necessary just to include already initialized matrix elements for performing the Min function inside the pseudocode above.
Once the matrix has been filled, a path of minimal cost can be calculated by tracing back through the DP matrix beginning from A[n-1,n-1], and ending in A[0,0] :
DisparityMapL [i,y] = j-i
DisparityMapR [j,y] = i-j
Up = A[i-1,j]
Left = A[i,j-1]
UpLeft = A[i-1,j-1]
Minimum = Min( Up,Left,UpLeft )
case Minimum of
UpLeft : i=i-1;j=j-1
Left : j=j-1
Up : i=i-1
end

For i = 0 to end,
For j = 0 to end,
...

Step 1: CALCULATE MATRIX
a1 = Right[i,y]
a2 = Left[j,y]
a3 = diff(a1,a2)
a4 = a3*scale + weight // denoise
a5 = f(a4, smooth[i,j])
b1 = A[i-1,j]
b2 = A[i,j-1]
b3 = A[i-1,j-1]
b4 = minimum(b1, b2, b3)
b5 = b4 - path[i,j] // Reusing paths
path[i,j] = path[i,j] * 0.875 // Reusing paths
A[i,j] = a5 + b5 // Write DP matrix

Step 2: FIND PATH
c1 = A[i-1,j]
c2 = A[i,j-1]
c3 = A[i-1,j-1]
c4 = minimum(c1, c2, c3)
c1 = c1 + weight1
c2 = c2 + weight2
c3 = c3 + weight3
// Choose new position
if (c4 <= c1) { j = j-1; i = i-1; } else
if (c4 <= c2) { j = j-1 } else
if (c4 <= c3) { i = i-1 }
// Remember paths
path[i,j] = constant
// Write disparity map
disparity[i,y] = j

One improvement we didn't implement: line skipping. (At first, every n’th horizontal line is calculated to find bounding space for possible disparities in between.)

\end{comment}


% vim:ft=tex
% rubber: module xelatex

\subsection{Face recognition}
\label{sec:face-rec}

We implemented PCA for 2D face recognition.

Face images are rectified, then normalised so that main face features (corners of the eyes and top of the mouth) are aligned to three points averaged over all the training images.

The normalised faces are cropped to a ratio of the main face feature positions. An ellipse is fitted to the bounding box of the new image frame. Anything outside the ellipse (background to the face) is replaced with zero values.

The normalised, cropped faces have stereo matching (dynamic programming) run on them.

We now have a database of training images and their associated depth maps.

Up to six channels of data from each image can be used: red values, green values, blue values, Hue values, Saturation values, and depth values. All the values from one pixel are put into a string of information, and all the pixels appended into a single image vector.

The covariance matrices for these image vectors are calculated. At this point, we began to use the opencv PCA functions to reduce the dimensionality of the space, so that we could work with larger images.

The orthonormal basis is determined. All training images are projected into the eigen space and then back again. Error between the original image and the reconstructed image (in terms of square of absolute distance) is calculated.

We can keep some quantity of image data by calculating the sum of eigenvalues for the components we are keeping, divided by the sum of all eigenvalues.


